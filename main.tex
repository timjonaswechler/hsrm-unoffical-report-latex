\documentclass[thesis]{.src/hsrmreport}
    % Allgemeine Informationen des Dokuments
        % \Fachbereich{}
        % \Studiengang{}
        % \Studienrichtung{}
        \Dokumentenart{Bachelorarbeit}

    % Informationen für eine Ausarbeitung / Gruppenarbeit 
        \Kurs{Die künsterlische Vielfalt}
        \Studierende{Ein Mensch mit vielen Namen}{000010}
        \Studierende{Meiner Wenigkeit}{111111}

    % Informationen für einen Thesis oder Praktikumsbericht
        \Autor{Meiner Wenigkeit}{<Mat.Nr.>}
        \Strasse{Hier-und-jetzt Straße 55}
        \PLZ{12345}
        \Wohnort{Traumschloss}

        \Beteiligte{Referent}{Prof. Dr. von Zuhause}
        \Beteiligte{Koreferent}{Prof. Dr. von Zuhause II}
        \Beteiligte{Betreuer}{Prof. Dr. von Zuhause III}

        \Unternehmen{TraumJOB GmbH}


\addbibresource{bib/bib-refs.bib}

\GlsXtrLoadResources[
    src={./bib/glossar},
    save-locations=false,
    % section=chapter,
    field-aliases={
        unit=symbol,
        formula=name,
        mathname=description,
        identifier=category,
    },
    entry-type-aliases={
        unit=symbol,
        math=symbol,
        abkuerzung=abbreviation
    },
    symbol-sort-fallback={name},
    type={same as category},
    set-widest
]

\begin{document}
    % Deckblatt
        \Deckblatt{Test}
        % \listoftodos
    % Eigenständigkeitserklärung
        \chapter*{Eigenständigkeitserklärung}
\pdfbookmark[0]{Eigenständigkeitserklärung}{Eigenständigkeitserklärung}
Hiermit versichere ich, dass ich die vorliegende Arbeit selbständig und
ohne unzulässige Hilfe Dritter verfasst habe.

Die aus fremden Quellen direkt oder indirekt übernommenen Texte,
Gedankengänge, Konzepte usw. in meinen Ausführungen habe ich als
solche eindeutig gekennzeichnet und mit vollständigen Verweisen auf
die jeweilige Urheberschaft und Quelle versehen.

Alle weiteren Inhalte wie Textteile, Abbildungen, Tabellen etc. ohne
entsprechende Verweise stammen im urheberrechtlichen Sinn von mir.

Die vorliegende Arbeit wurde bisher weder im In- noch im Ausland in
gleicher oder ähnlicher Form einer anderen Prüfungsbehörde vorgelegt.

Mir ist bekannt, dass ein Täuschungsversuch vorliegt, wenn sich eine
der vorstehenden Versicherungen als unrichtig erweist.

\vspace{20mm}

Hiermit erkläre ich mein Einverständnis mit den im Folgenden aufgeführten Verbreitungsformen \hl{dieser} \insertDokumentenart \todo[info]{Text überprüfen.}:

\begin{center}
  \begin{tabular}{ | l | c | c | }
    \hline
    \textbf{Verbreitungsform}                                             & \textbf{ja} & \textbf{nein} \\ \hline
    Einstellung der Arbeit in die Hochschulbibliothek mit Datenträger     & X           &               \\ \hline
    Einstellung der Arbeit in die Hochschulbibliothek ohne Datenträger    & X           &               \\ \hline
	Veröffentlichung des Titels der Arbeit im Internet                    & X           &               \\ \hline
	Veröffentlichung der Arbeit im Internet                               & X           &               \\
    \hline
  \end{tabular}
  \todo[info,inline]{Inhalt der Tabelle gegebenfalls an die aktuelle Richtlinien anpassen.}
\end{center}

\vfill

\begin{tabular}{p{7.5cm}}
    \hrulefill \\
    \Autorname 
    \insertAbgabeOrt, \insertDatum \\
    Hochschule RheinMain
\end{tabular}


    
    % Sperrvermerk
        \chapter*{Sperrvermerk}
\pdfbookmark[0]{Sperrvermerk}{Sperrvermerk}
Die vorliegende \insertDokumentenart beinhaltet vertrauliche
Informationen und Daten des Unternehmens \insertUnternehmen.
Diese \insertDokumentenart darf ausschließlich vom Referenten,
Korreferenten und dem Prüfungsausschuss eingesehen
werden. Eine Vervielfältigung und Veröffentlichung der
\insertDokumentenart ist auch auszugsweise nicht erlaubt.
Dritten darf diese Arbeit nur mit der ausdrücklichen
Genehmigung des Verfassers und des Unternehmens
zugänglich gemacht werden.

    % Abstrakt    
        \chapter*{Abstrakt}
\pdfbookmark[0]{Abstrakt}{Abstrakt}

    
    % Verzeichnisse
        % % WICHTIGER HINWEIS: 
% wenn man ein Abkürzungsverzeichnis oder wie in diesem Beispiel mehrere in seinem Dokument mit einbinden möchte so muss 
% die Routine mit der das Dokument erstellt wird angepasst werden. 
% Bei der Verwendung dieser Dokumentenklasse:
%  - pdflatex
%  - bib2gls
%  - pdflatex
%  - pdflatex

% Dem Befehl bib2gls muss das Argument --group noch hinzugefügt werden. 


\chapter*{Begriffsverzeichnis}
\pdfbookmark[0]{Begriffsverzeichnis}{Begriffsverzeichnis}

\pdfbookmark[1]{Abkürzungsverzeichnis}{Abkürzungsverzeichnis}
\printunsrtglossary[type=abkuerz,style=index]

\pdfbookmark[1]{Begriffserklärung}{Begriffserklärung}
\printunsrtglossary[type=erklaer,style=altlist]


\chapter*{Verwendete Symbole}
\pdfbookmark[0]{Verwendete Symbole}{Verwendete Symbole}
\printunsrtglossary[type=math,style=mcolalttree]
{%
    \renewcommand*{\glstreenamefmt}[1]{#1}%
    \renewcommand*{\glstreegroupheaderfmt}[1]{\textbf{#1}}%
}
\pdfbookmark[1]{Mathematische Symbole}{Mathematische Symbole}

\printunsrtglossary[type=units,style=units]
\pdfbookmark[1]{Physikalsiche Größen}{Physikalsiche Größen}


    % Inhaltsverzeichnis
        \begingroup		
            \pagestyle{plain}
            \tableofcontents
            \cleardoublepage
        \endgroup
        \newpage

    % Normale Seitenzahlen    
        \pagenumbering{arabic}

    % Das Seitenlayout mit Kapitel und Unterkapitel im Header jeder Seite des Berichts
        \pagestyle{scrheadings}
       

    % Literaturverzeichnis:
    % Listet alle verwendeten Quellen auf, einschließlich 
    %       wissenschaftlicher Artikel, Bücher, Dissertationen, etc.
        % WICHTIGER HINWEIS: 
% wenn man ein Quellenverzeichnis in seinem Dokument mit einbinden möchte so muss 
% die Routine mit der das Dokument erstellt wird angepasst werden. 
% Bei der Verwendung dieser Dokumentenklasse:
%  - pdflatex
%  - biber
%  - pdflatex
%  - pdflatex


% Möchte man nur ein einfaches Quellenverzeichnis realisieren, dann geht das mit folgendem Befehl
\printbibliography

% Es gibt auch die Möglichkeit ein Quellenverzeichnis auf zu teilen.

% Im folgenden wird ein Quellenverzeichnis was unterteilt einem in Quellen für Kaufteile und Quellen der Recherche erstellt.
% Für dieses Beispiel wird mit dem keyword = "Kaufteile" eine Auswahl getroffen.
% Wenn der Eintrag in der bib-Datei dieses keyword besitzt wird dieser dem Quellenverzeichnis für Kaufteile aufgelistet. 
% Besitzt er dieses keyword nicht taucht der Eintrag in dem anderen Quellenverzeichnis auf.
%   @entry{label,
%     author = {},
%     title  = {},
%     keywords = {Kaufteile}

\printbibheading
\printbibliography[
    keyword=Kaufteile,
    heading=subbibliography,%
    title={Quellen der Kaufteile},
]
\printbibliography[
    notkeyword=Kaufteile,
    heading=subbibliography,%
    title={Quellen der Recherche},
]

 

    % Abbildungsverzeichnis
        % \begingroup		
        %     \pagestyle{plain}
        %     \listoffigures
        %     \cleardoublepage
        % \endgroup
            
    % Tabellenverzeichnis
        % \begingroup		
        %     \pagestyle{plain}
        %     \listoftables
        %     \cleardoublepage
        % \endgroup
                
    % Codeverzeichnis
        % \begingroup		
        %     \pagestyle{plain}
        %     \lstlistoflistings
        %     \cleardoublepage
        % \endgroup

    \appendix
        % - Vollständiger Satz der Zeichnungen
        % - Datenblätter verwendeter Norm- und Zukaufteile
        % - Datenblätter verwendeter Werkstoffe
        % - weitere Informationen      
        
\end{document}