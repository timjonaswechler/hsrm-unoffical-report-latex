% WICHTIGER HINWEIS: 
% wenn man ein Quellenverzeichnis in seinem Dokument mit einbinden möchte so muss 
% die Routine mit der das Dokument erstellt wird angepasst werden. 
% Bei der Verwendung dieser Dokumentenklasse:
%  - pdflatex
%  - biber
%  - pdflatex
%  - pdflatex


% Möchte man nur ein einfaches Quellenverzeichnis realisieren, dann geht das mit folgendem Befehl
\printbibliography

% Es gibt auch die Möglichkeit ein Quellenverzeichnis auf zu teilen.

% Im folgenden wird ein Quellenverzeichnis was unterteilt einem in Quellen für Kaufteile und Quellen der Recherche erstellt.
% Für dieses Beispiel wird mit dem keyword = "Kaufteile" eine Auswahl getroffen.
% Wenn der Eintrag in der bib-Datei dieses keyword besitzt wird dieser dem Quellenverzeichnis für Kaufteile aufgelistet. 
% Besitzt er dieses keyword nicht taucht der Eintrag in dem anderen Quellenverzeichnis auf.
%   @entry{label,
%     author = {},
%     title  = {},
%     keywords = {Kaufteile}

\printbibheading
\printbibliography[
    keyword=Kaufteile,
    heading=subbibliography,%
    title={Quellen der Kaufteile},
]
\printbibliography[
    notkeyword=Kaufteile,
    heading=subbibliography,%
    title={Quellen der Recherche},
]

