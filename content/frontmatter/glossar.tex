% WICHTIGER HINWEIS: 
% wenn man ein Abkürzungsverzeichnis oder wie in diesem Beispiel mehrere in seinem Dokument mit einbinden möchte so muss 
% die Routine mit der das Dokument erstellt wird angepasst werden. 
% Bei der Verwendung dieser Dokumentenklasse:
%  - pdflatex
%  - bib2gls
%  - pdflatex
%  - pdflatex

% Dem Befehl bib2gls muss das Argument --group noch hinzugefügt werden. 


\chapter*{Begriffsverzeichnis}
\pdfbookmark[0]{Begriffsverzeichnis}{Begriffsverzeichnis}

\pdfbookmark[1]{Abkürzungsverzeichnis}{Abkürzungsverzeichnis}
\printunsrtglossary[type=abkuerz,style=index]

\pdfbookmark[1]{Begriffserklärung}{Begriffserklärung}
\printunsrtglossary[type=erklaer,style=altlist]


\chapter*{Verwendete Symbole}
\pdfbookmark[0]{Verwendete Symbole}{Verwendete Symbole}
\printunsrtglossary[type=math,style=mcolalttree]
{%
    \renewcommand*{\glstreenamefmt}[1]{#1}%
    \renewcommand*{\glstreegroupheaderfmt}[1]{\textbf{#1}}%
}
\pdfbookmark[1]{Mathematische Symbole}{Mathematische Symbole}

\printunsrtglossary[type=units,style=units]
\pdfbookmark[1]{Physikalsiche Größen}{Physikalsiche Größen}
